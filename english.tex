\documentclass{resume}
\usepackage{latexsym}
\usepackage{url}
\usepackage{hyperref}
\hypersetup{
    colorlinks=true,
    linkcolor=blue,
    filecolor=magenta,
    urlcolor=cyan,
}

\renewcommand{\categoryfont}{\sc}

%
% set the space used for category titles here:
% use the same value for oddsidemargin and marginparwidth [the latter 
% 		will be reset to account for marginparsep]
% 
\setlength{\oddsidemargin}{1in}
\setlength{\marginparwidth}{1in}
% 
% calculate other dimensions [textwidth and evensidemargin] 
% in function of oddsidemargin and marginparwidth: 
% would be nicer to put in the class file...
%
\addtolength{\marginparwidth}{-\marginparsep}
\setlength{\evensidemargin}{\oddsidemargin}
\setlength{\textwidth}{\paperwidth}
\addtolength{\textwidth}{-2in}
\addtolength{\textwidth}{-2\oddsidemargin}
\addtolength{\textwidth}{\marginparwidth}
\addtolength{\textwidth}{\marginparsep}

%
\setlength{\topmargin}{0in}
%
%
\renewcommand{\labelcitem}{$\diamond$}
\renewcommand{\labelitemi}{$\cdot$}
\newcommand{\first}{$1^{\mbox{\scriptsize st}}$\ }
\newcommand{\second}{$2^{\mbox{\scriptsize nd}}$\ }
\newcommand{\third}{$3^{\mbox{\scriptsize rd}}$\ }

\author{Leonardo Sandoval}

% ------ Address --------------------------------------------------------

\address{}
        {Paseo del Cerezo 15, \\
         Colonia Hacienda La Herradura\\
         Zapopan, Jalisco, M\'exico \\
         ZIP 45226 \\
         (+52) 33 12 67 90 39\\
         \mbox{\small\tt leo.san.gon@gmail.com}}

\def\udg{University of Guadalajara}
\def\tub{\udg}
\def\inaoe{Instituto Nacional de Astrof\'{\i}sica, \'Optica y Electr\'onica}
\def\itesm{Tecnol\'ogico de Monterrey campus Monterrey}
\def\facty{Faculty of Computer Science}
\def\linaro{Linaro Limited}
\def\intel{Intel Corporation}
\def\cl{Clear Linux OS}
\def\fsl{Freescale Semiconductor}

\begin{document}
\maketitle

% ------- Education ---------------------------------------------------

\begin{category}{Education}
\citem{\inaoe}\\
MS in Computer Science. Thesis defended in January 2005.

\citem{\udg}\\
Diploma in Computer Science, August 1998.
\citem{Scholarships}
\begin{itemize}
\item Mexican-French Scientific Agreement (ANUIES-ECOS). January -- December 2003.
\item Mexican Science Sponsor (CONACYT). January 2001 -- December 2002.
\end{itemize}
\end{category}

% -------- Work experience --------------------------------------------

\begin{category}{Work \\experience}

\citem{\intel}\\
Feb 15 -- May 20
\begin{itemize}
\item \textbf{Cloud Software Engineer} (Oct 18 -- May 20)\\
  Member of the \href{https://github.com/intel/stacks}{Software Stacks} team,
  providing specialized and optimized \href{https://clearlinux.org/}{\cl} Reference Stacks.
  Leo is one of the Integration Architects who has lead two stacks:
  the \href{https://github.com/intel/stacks/blob/master/dars/dars.rst}{Data Analytic}
  and \href{https://github.com/intel/stacks/blob/master/mers/README.md}{Media} Rererence Stacks,
  from design to release.
\end{itemize}

\begin{itemize}
\item \textbf{Performance Software Engineer} (Dec 2017 -- Oct 2018)\\
  Member of the Clear Linux \href{https://clearlinux.org/}{Performance Team}.
  Activities include: Performance Package Monitoring and Routine Optimization.
  For the latter, Leo's contributions focused on
  \href{https://www.gnu.org/software/libc/}{GNU C library},
  adopting \href{}{AVX2} technology for \textit{strcmp} and \textit{strcpy} string routines.
  See \href{https://patchwork.ozlabs.org/project/glibc/list/?submitter=74072}{patchwork}
  for completed work.
\end{itemize}

\begin{itemize}
\item \textbf{Embedded Software Engineer} (February 2015 -- December 2017)\\
  Member of the Core \href{https://www.yoctoproject.org/}{Yocto Project} Team,
  providing support for several sub-projects:
  \href{https://lists.openembedded.org/g/bitbake-devel/search?p=recentpostdate%252Fsticky%2C%2C%2C20%2C2%2C0%2C0&q=Leonardo+Sandoval}{bitbake},
  \href{https://lists.openembedded.org/g/openembedded-core/search?p=recentpostdate%252Fsticky%2C%2C%2C20%2C2%2C0%2C0&q=Leonardo+Sandoval}{OpenEmbedded-Core} and
    \href{https://lists.yoctoproject.org/g/poky/search?p=created%2C0%2C%2C1%2C2%2C0%2C0&q=Leonardo+Sandoval}{Poky}. Initial maintainer of
      \href{http://git.yoctoproject.org/cgit/cgit.cgi/patchtest/}{Patchtest},
    a framework and test suite for \href{https://patchwork.openembedded.org/project/oe-core/series/?ordering=-last_updated}{Open-embedded-Core Patches}.
\end{itemize}

\citem{\fsl}\\
Jul 2012 -- Feb 2015
\begin{itemize}
\item \textbf{Embedded Software Engineer - Professional Services Consultant} (Apr 14 -- Feb 15)\\
  Board Support Package (BSP) development and support for \href{https://www.nxp.com/products/processors-and-microcontrollers/arm-processors/i-mx-applications-processors:IMX_HOME}{i.MX}
  Multimedia Processors. Direct support to US customers for the entire Software Stack: from boot loader up to the user-space.
\end{itemize}

\begin{itemize}
\item \textbf{Embedded Software Engineer - Field Application Engineer} (Jul 12 -- Apr 14)\\
  Customer support for customers located in the USA Central Region and M\'exico using
  \href{https://www.nxp.com/products/processors-and-microcontrollers/arm-processors/i-mx-applications-processors:IMX_HOME}{i.MX}
  Multimedia Processors, with main focus on issues raised from \href{https://git.yoctoproject.org/cgit/cgit.cgi/meta-fsl-arm}{meta-fsl-arm} BSP meta-layer.
  Key \href{https://community.nxp.com/t5/user/viewprofilepage/user-id/25586}{member}
  of the Linux build system migration, supporting community users to transition from \textit{ltib} to the Yocto Project.
\end{itemize}

\citem{Texas Instruments - Dextra Technologies}\\
August 2006 -- June 2011
\begin{itemize}
\item \textbf{Embedded Software Engineer} (August 2007 -- June 2011).\\
  Creating and maintenance of GStreamer plugins for the \texttt{OMAP} family
  processors. Plugins based on TI \texttt{OpenMAX} IL multimedia interface.
  Maintainer of the camera and video encoder plugins.
\item \textbf{Embedded Software Engineer} (August 2006 -- August 2007)\\
  Software developer for the \texttt{OpenMAX} IL Camera component for
  \texttt{OMAP2} family processors.
\end{itemize}
\end{category}

\begin{category}{Academic \\experience}

\citem\itesm\\
\begin{itemize}
\item \textbf{Software Embedded} (August 2014 -- November 2014).\\
  A \textit{Embedded Linux Software} course imparted on the Department of
  Master of Electronic. Topics covered: cross-toolchain, bootloader,
  kernel, filesystems and multimedia through HW accelerators using
  \texttt{i.MX} processors.
\end{itemize}
\end{category}


\begin{category}{Research \\experience}

\citem\itesm\\
\begin{itemize}
\item \textbf{Web programmer} (Part time, August 2009 -- January 2011).\\
  Design and implementation of an Second factor authentication system using
  browser User Agent. A functional prototype built and delivered to customer.
\item \textbf{Web programmer} (Full time, August 2007 -- July 2008).\\
  Design and implementation of a classification method for detecting intruders
  using keystrokes typing rhythm. The designed algorithm documented and
  delivered successfully to customer (Google).
\end{itemize}

\citem{\inaoe}\\
\begin{itemize}
\item \textbf{Master thesis in Computer Astrophysics} (January 2001 -- December 2004).\\
  Comparison between Decay Times of Satellite Galaxies using $N$-body
  numerical simulations and a two-body motion equation with dynamical
  friction. Several galaxy density profiles were compared. Thesis partially done in Marseille, France\\
  Advisors:  Ivanio Puerari (INAOE, M\'exico) and Lia Athanassoula (Observatoire de Marseille, France).
\item \textbf{Scientific Programmer} (August -- December 2000).
  Comparison between two classification methods (\textsl{k}-nn and neural networks) using stellar spectra.\\
  Advisor: Olac Fuentes (INAOE, M\'exico).
\end{itemize}
\end{category}

% -------- Publication --------------------------------------------

\begin{category}{Talks}

\citembullet Simulation of Decay Times of Satellite Galaxies using Semi-Analytical methods.\\
Proceedings of The National Convention of Astronomy 2003.\\
Instituto de Astronom\'{\i}a, UNAM. \\
Authors: Leonardo Sandoval, Lia Athanassoula and Jorge Villa
\end{category}

% ------- Skills ------------------------------------------------------
\begin{category}{Skills}
\citembullet Programming languages: Python, shell (bash), C, X86 assembler
\citembullet Version controllers: Git
\citembullet Linux Building Systems: LTIB, Yocto Project
\end{category}

\begin{category}{Languages}
\citembullet Native Spanish
\citembullet Fluent spoken/written English
\citembullet Fair \textbf{German}. ZDaF (Zertifikat Deutsch als Fremdsprache) obtained in Goethe-Institut, Guadalajara, Jalisco, Mexico (1996-1998)
\end{category}

% ----- Hobbies ------------------------
\begin{category}{Hobbies}
\citembullet Mountain climbing, travelling
\citembullet Tech book reading: x86 architecture, compilers, vectorization, binary analytics
\citembullet Meditation: Mindfulness, Yoga, Kriya Yoga
\end{category}

% -------- Reference --------------------------------------------
\begin{category}{Reference} 
\citemnobullet Available on request.
\end{category}

\end{document}
