\documentclass{resume}
\usepackage{latexsym}
\usepackage{url}
\usepackage{hyperref}
\hypersetup{
    colorlinks=true,
    linkcolor=blue,
    filecolor=magenta,
    urlcolor=cyan,
}

\renewcommand{\categoryfont}{\sc}

%
% set the space used for category titles here:
% use the same value for oddsidemargin and marginparwidth [the latter 
% 		will be reset to account for marginparsep]
% 
\setlength{\oddsidemargin}{1in}
\setlength{\marginparwidth}{1in}
% 
% calculate other dimensions [textwidth and evensidemargin] 
% in function of oddsidemargin and marginparwidth: 
% would be nicer to put in the class file...
%
\addtolength{\marginparwidth}{-\marginparsep}
\setlength{\evensidemargin}{\oddsidemargin}
\setlength{\textwidth}{\paperwidth}
\addtolength{\textwidth}{-2in}
\addtolength{\textwidth}{-2\oddsidemargin}
\addtolength{\textwidth}{\marginparwidth}
\addtolength{\textwidth}{\marginparsep}

%
\setlength{\topmargin}{0in}
%
%
\renewcommand{\labelcitem}{$\diamond$}
\renewcommand{\labelitemi}{$\cdot$}
\newcommand{\first}{$1^{\mbox{\scriptsize st}}$\ }
\newcommand{\second}{$2^{\mbox{\scriptsize nd}}$\ }
\newcommand{\third}{$3^{\mbox{\scriptsize rd}}$\ }

\author{Leonardo Sandoval Gonzalez}

% ------ Address --------------------------------------------------------

\address{}
        {Rinconada Vallarta ext 4 int 2, \\
         Colonia Ciudad Granja\\
         Zapopan, Jalisco, M\'exico \\
         ZIP 45010 \\
         (+52) 33 12 67 90 39\\
         \mbox{\small\tt leo.san.gon@gmail.com}}

\def\udg{University of Guadalajara}
\def\tub{\udg}
\def\inaoe{Instituto Nacional de Astrof\'{\i}sica, \'Optica y Electr\'onica}
\def\itesmmty{Tecnol\'ogico de Monterrey campus Monterrey}
\def\itesmgdl{Tecnol\'ogico de Monterrey campus Guadalajara}
\def\iteso{Instituto Tecnol\'ogio y de Estudios Superiores de Occidente}
\def\facty{Faculty of Computer Science}
\def\linaro{Linaro Limited}
\def\intel{Intel Corporation}
\def\cl{ClearLinux}
\def\fsl{Freescale Semiconductor}
\def\texasins{Texas Instruments}
\def\dextra{Dextra Technologies}

\begin{document}
\maketitle

% ------- Education ---------------------------------------------------

\begin{category}{Education}

  \citem{\inaoe}\\
  MS in Computer Science. Thesis defended in January 2005.

  \citem{\udg}\\
  Diploma in Computer Science, August 1998.
  \citem{Scholarships}
  \begin{itemize}
  \item Mexican-French Scientific Agreement (ANUIES-ECOS). January -- December 2003.
  \item Mexican Science Sponsor (CONACYT). January 2001 -- December 2002.
  \end{itemize}

\end{category}

% -------- Work experience --------------------------------------------

\begin{category}{Work \\experience}

  \citem{\linaro}\\
  May 2020 -- August 2023

  \begin{itemize}
  \item \textbf{Senior Software Engineer - CodeLinaro} (Apr 23 -- Aug 23)\\
    Part of the core \href{www.codelinaro.org}{Codelinaro} team, lead the Git and CI/CD Repository migration
    from \href{https://git.linaro.org}{Linaro Git} / \href{https://ci.linaro.org}{Linaro Jenkins} to
    \href{https://git.codelinaro.org}{CodeLinaro Git}, providing migration tools, documentation and support to
    internal Linaro teams.
  \end{itemize}

  \begin{itemize}
  \item \textbf{Senior Software Engineer - Qualcomm} (Jan 22 -- Apr 23)\\
    Part of the Qualcomm Landing Team CI team, lead the CI/CD Migration from \\
    \href{https://ci.linaro.org/view/qclt/}{Linaro Qualcomm Jenkins} into
    \href{https://git.codelinaro.org/linaro/qcomlt/ci/configs}{CodeLinaro CI}, the latter becoming the official CI/CD environment.
    The Qualcomm Landing team members are \href{https://kernelnewbies.org/DevelopmentStatistics}{top contributors} on the Linux project,
    where the CI/CD plays a key role on testing patches before becoming public.
  \end{itemize}

  \begin{itemize}
  \item \textbf{Senior Software Engineer - MbedTLS} (Oct 21 -- Jan 22)\\
    Lead the CI migration from \href{https://www.arm.com/}{ARM} internal Jenkins instance into a public \href{https://github.com/ARMmbed/mbedtls}{Mbed TLS}.
  \end{itemize}

  \begin{itemize}
  \item \textbf{Senior Software Engineer - Trusted Firmware-A} (May 20 -- Oct 21)\\
    Lead the CI migration of the \href{https://www.trustedfirmware.org/projects/tf-a/}{TF-A}
    project from an \href{https://www.arm.com/}{ARM} internal Jenkins instance into a public instance \href{https://ci.trustedfirmware.org/}{TF-A CI}.
    Work found at \href{https://review.trustedfirmware.org/q/owner:leonardo.sandoval%2540linaro.org}{patch review page}.
  \end{itemize}

  \citem{\intel}\\
  February 2015 -- May 2020

  \begin{itemize}
  \item \textbf{Integration Architect - Intel Stacks} (Oct 18 -- May 20)\\
    Member of the \href{https://github.com/intel/stacks}{Software Stacks} team,
    providing specialized and optimized \href{https://clearlinux.org/}{\cl} Software Reference Stacks.
    Lead two stacks as integration architect:
    \href{https://github.com/intel/stacks/blob/master/dars/dars.rst}{Data Analytic}
    and Media, from design to release.
  \end{itemize}

  \begin{itemize}
  \item \textbf{Performance Engineer - ClearLinux} (Dec 2017 -- Oct 2018)\\
    Member of the \href{https://clearlinux.org/}{ClearLinux} Performance Team.
    Activities included: Performance Package performance Monitoring and Low-level Routine Optimization.
    For the latter, contributions focused on the \href{https://www.gnu.org/software/libc/}{GNU C library},
    adopting the \href{https://patchwork.ozlabs.org/project/glibc/list/?submitter=74072}{AVX2} technology
    for \textit{strcmp} and \textit{strcpy} string routines.
  \end{itemize}

  \begin{itemize}
  \item \textbf{Senior Embedded Software Engineer - Yocto Project} (February 2015 -- December 2017)\\
    Member of the Core \href{https://www.yoctoproject.org/}{Yocto Project} Team,
    providing support for several sub-projects:
    \href{https://lists.openembedded.org/g/bitbake-devel/search?p=recentpostdate%252Fsticky%2C%2C%2C20%2C2%2C0%2C0&q=Leonardo+Sandoval}{bitbake},
    \href{https://lists.openembedded.org/g/openembedded-core/search?p=recentpostdate%252Fsticky%2C%2C%2C20%2C2%2C0%2C0&q=Leonardo+Sandoval}{OpenEmbedded-Core} and
    \href{https://lists.yoctoproject.org/g/poky/search?p=created%2C0%2C%2C1%2C2%2C0%2C0&q=Leonardo+Sandoval}{Poky}. Initial maintainer of
    \href{https://git.yoctoproject.org/cgit/cgit.cgi/patchtest/}{Patchtest},
    a framework and test suite for OpenEmbedded Core Patches.
  \end{itemize}

  \citem{\fsl}\\
  July 2012 -- February 2015

  \begin{itemize}
  \item \textbf{Senior Embedded Software Engineer - Professional Services} (Apr 14 -- Feb 15)\\
    Board Support Package (BSP) development and support for
    \href{https://www.nxp.com/products/processors-and-microcontrollers/arm-processors/i-mx-applications-processors:IMX_HOME}{i.MX}
    Multimedia Processors. Direct support to US customers for the entire Software Stack: from bootloader up to the user-space, full Linux
    stack.
  \end{itemize}

  \begin{itemize}
  \item \textbf{Senior Embedded Software Engineer - Field Application Engineer} (Jul 12 -- Apr 14)\\
    Customer support for customers located in the USA Central Region and M\'exico using
    \href{https://www.nxp.com/products/processors-and-microcontrollers/arm-processors/i-mx-applications-processors:IMX_HOME}{i.MX}
    Multimedia Processors, with main focus on issues raised from \href{https://git.yoctoproject.org/cgit/cgit.cgi/meta-fsl-arm}{meta-fsl-arm} BSP meta-layer.
    Key \href{https://community.nxp.com/t5/user/viewprofilepage/user-id/25586}{member}
    of the Linux build system migration, supporting community users to transition from \textit{ltib} to the Yocto Project.
  \end{itemize}

  \citem{\dextra - \texasins}\\
  August 2006 -- June 2011

  \begin{itemize}
  \item \textbf{Embedded Software Engineer - GStreamer} (August 2007 -- June 2011).\\
    Camera and Video Encoder GStreamer plugin maintainer for the \texttt{OMAP} family
    processors. Plugins based on TI \texttt{OpenMAX} IL multimedia middle-ware layer.
  \end{itemize}

  \begin{itemize}
  \item \textbf{Embedded Software Engineer - OpenMAX} (August 2006 -- August 2007)\\
    Software developer for the \texttt{OpenMAX} IL Camera component for
    \texttt{OMAP2} family processors.
  \end{itemize}
  
\end{category}

\begin{category}{Academic \\Experience}

  \citem{\iteso}\\
  August 2023 -- Present
  \begin{itemize}
  \item \textbf{Operating Systems} (August 2023 -- ).\\
    Currently teaching a \textit{Fundamentals of Operating System} course at the Department of Electronics,
    Systems and Informatics for undergraduate students.
  \end{itemize}

  \citem{\itesmgdl}\\
  August 2014 -- November 2014
  \begin{itemize}
  \item \textbf{Embedded Linux Course} (August 2014 -- November 2014).\\
    A \textit{Embedded Linux Course} course imparted on the Department of
    Master of Electronic, covering all SW stack on a embedded Linux device
  \end{itemize}

\end{category}

\begin{category}{Research \\experience}

  \citem{\itesmmty}\\
  August 2007 -- January 2011

  \begin{itemize}
  \item \textbf{Web programmer - Computer Science Department} (Part time, August 2009 -- January 2011).\\
    Design and implementation of an Second factor authentication system using
    browser User Agent. A functional prototype built and delivered to customer.

  \item \textbf{Web programmer - Computer Science Department} (Full time, August 2007 -- July 2008).\\
    Design and implementation of a classification method for detecting intruders
    using keystrokes typing rhythm. The designed algorithm documented and
    delivered successfully to customer (Google).
  \end{itemize}
  
  \citem{\inaoe}\\
  August 2000 -- December 2004

  \begin{itemize}
  \item \textbf{Master thesis in Computer Astrophysics} (January 2001 -- December 2004).\\
    Comparison between Decay Times of Satellite Galaxies using $N$-body
    numerical simulations and a two-body motion equation with dynamical
    friction. Several galaxy density profiles were compared. Thesis partially done in Marseille, France\\
    Advisors:  Ivanio Puerari (INAOE, M\'exico) and Lia Athanassoula (Observatoire de Marseille, France).

  \item \textbf{Scientific Programmer - Machine Learning} (August -- December 2000).
    Comparison between two classification methods (\textsl{k}-nn and neural networks) using stellar spectra.\\
    Advisor: Olac Fuentes (INAOE, M\'exico).
  \end{itemize}
\end{category}

% -------- Publication --------------------------------------------

\begin{category}{Talks}

\citembullet Simulation of Decay Times of Satellite Galaxies using Semi-Analytical methods.\\
Proceedings of The National Convention of Astronomy 2003.\\
Instituto de Astronom\'{\i}a, UNAM. \\
Authors: Leonardo Sandoval, Lia Athanassoula and Jorge Villa
\end{category}

% ------- Skills ------------------------------------------------------
\begin{category}{Skills}
  \citembullet Open Source development under the Linux ecosystem
  \citembullet Programming languages: Python, Bash, C, X86-64/ARCH64 assembler
  \citembullet CI/CD: Jenkins, Gitlab, Github
  \citembullet Linux Building Systems: bitbake (YP), buildroot
  \citembullet Linux Command Line Tools
\end{category}

\begin{category}{Languages}
  \citembullet Native Spanish
  \citembullet Fluent spoken/written English
  \citembullet Fair \textbf{German}. ZDaF (Zertifikat Deutsch als Fremdsprache) obtained in Goethe-Institut, Guadalajara, Jalisco, Mexico (1996-1998)
\end{category}

% ----- Hobbies ------------------------
\begin{category}{Hobbies}
  \citembullet Hiking, traveling, cooking, swimming, running, functional, amateur chess player
  \citembullet Reading: Technology books (array languages, system performance, low-leve vectorization, parallel programming)
  \citembullet Meditation: Mindfulness
\end{category}

% -------- Reference --------------------------------------------
\begin{category}{Reference} 
  \citemnobullet Available on request.
\end{category}

\end{document}
