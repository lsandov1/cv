\documentclass{resume}
\usepackage{latexsym}
\usepackage{url}
\usepackage{hyperref}

\renewcommand{\categoryfont}{\sc}

%
% set the space used for category titles here:
% use the same value for oddsidemargin and marginparwidth [the latter 
% 		will be reset to account for marginparsep]
% 
\setlength{\oddsidemargin}{1in}
\setlength{\marginparwidth}{1in}
% 
% calculate other dimensions [textwidth and evensidemargin] 
% in function of oddsidemargin and marginparwidth: 
% would be nicer to put in the class file...
%
\addtolength{\marginparwidth}{-\marginparsep}
\setlength{\evensidemargin}{\oddsidemargin}
\setlength{\textwidth}{\paperwidth}
\addtolength{\textwidth}{-2in}
\addtolength{\textwidth}{-2\oddsidemargin}
\addtolength{\textwidth}{\marginparwidth}
\addtolength{\textwidth}{\marginparsep}

%
\setlength{\topmargin}{0in}
%
%
\renewcommand{\labelcitem}{$\diamond$}
\renewcommand{\labelitemi}{$\cdot$}
\newcommand{\first}{$1^{\mbox{\scriptsize st}}$\ }
\newcommand{\second}{$2^{\mbox{\scriptsize nd}}$\ }
\newcommand{\third}{$3^{\mbox{\scriptsize rd}}$\ }

\author{Leonardo Sandoval}

% ------ Address --------------------------------------------------------

\address{}
        {Ciudad Quer\'etaro 211, Colonia M\'exico\\
         Zapopan, Jalisco, M\'exico \\
         home   (+52) 33 38 13 05 63\\
         mobil  (+52) 33 12 67 90 39\\
         \mbox{\small\tt leo.san.gon@gmail.com}}

\def\udg{University of Guadalajara}
\def\tub{\udg}
\def\inaoe{Instituto Nacional de Astrof\'{\i}sica, \'Optica y Electr\'onica}
\def\itesm{Tecnol\'ogico de Monterrey campus Monterrey}
\def\facty{Faculty of Computer Science}

\begin{document}
\maketitle

% ------- Education ---------------------------------------------------

\begin{category}{Education}
\citem{\inaoe}\\
MS in Computer Science. Thesis defended in January 2005.

\citem{\udg}\\
Diploma in Computer Science, August 1998.
\citem{Scholarships}
\begin{itemize}
\item Mexican-French Scientific Agreement (ANUIES-ECOS). January -- December 2003.
\item Mexican Science Sponsor (CONACYT). January 2001 -- December 2002.
\end{itemize}
\end{category}

% -------- Work experience --------------------------------------------

\begin{category}{Work \\experience}

\citem{Intel}\\

\begin{itemize}
\item \textbf{Software Engineer} (October 2018 -- present)\\
  Member of the \textbf{Software Stacks} team, providing specialized and optimized
  Clear Linux reference Docker Images. Reference images can be found at
  \href{https://github.com/clearlinux/dockerfiles}{Clear Linux
    dockerfiles}. Leo's participation focused on
  \href{https://github.com/clearlinux/dockerfiles/tree/master/stacks/dlrs}{Deep 
   Learning Reference Stacks}.
\end{itemize}


\begin{itemize}
\item \textbf{Software Engineer} (December 2017 -- October 2018)\\
  Member of the \textbf{Power and Performance} team which is part of the Clear
  Linux team. Leo's contributions focused on \textit{GNU
    C library}, adopting AVX2 technology for \textit{strcmp} and
  \textit{strcpy} string routines. Leo's contribution to the latter project can be found at
  \href{https://patchwork.ozlabs.org/project/glibc/list/?submitter=74072}{GNU
    C Library patchwork}.
\end{itemize}

\begin{itemize}
\item \textbf{Embedded Software Engineer} (February 2015 -- December 2017)\\
  Member of the core \href{https://www.yoctoproject.org/}{Yocto Project} Team
  providing support on \href{http://cgit.openembedded.org/openembedded-core/}{openembedded-core}
  and \href{http://cgit.openembedded.org/bitbake/}{bitbake} projects. Leo's contribution to 
  the openembedded-core project can be found at
  \href{https://patchwork.openembedded.org/project/oe-core/series/?ordering=-last_updated}{OE-Core patchwork}. Main maintainer of
  \href{http://git.yoctoproject.org/cgit/cgit.cgi/patchtest/}{patchtest}, framework and test suite for openembedded-core 
  community patches.
\end{itemize}

\citem{Freescale}\\
July 2012 -- February 2015
\begin{itemize}
\item \textbf{Embedded Software Engineer - Professional Services Consultant} (April 2014 -- February 2015)\\
  Board Support Package (BSP) development and support for \textbf{\texttt{i.MX}}
  Multimedia Processors. Direct support to US customers for all Software
  Stack, from boot loader to user space applications.
\end{itemize}

\begin{itemize}
\item \textbf{Embedded Software Engineer - Field Application Engineer} (July 2012 -- April 2014)\\
  Customer support for clients located in the USA Central Region and M\'exico using
  \texttt{i.MX} Multimedia Processors, with main focus on issues raised from software
  coming from the \textbf{\texttt{meta-fsl-arm}} layer, the openembeded BSP layer for
  the \texttt{i.MX}. During this period, the company
  moved from a in-house Linux Distribution builder (\texttt{ltib}) to the Yocto Project
  where Leonardo played a fundamental role in the
  \href{https://community.nxp.com/people/LeonardoSandovalGonzalez}{community}.
\end{itemize}

\citem{Texas Instruments - Dextra Technologies}\\
August 2006 -- June 2011
\begin{itemize}
\item \textbf{Embedded Software Engineer} (August 2007 -- June 2011).\\
  Creating and maintenance of GStreamer plugins for the \texttt{OMAP} family
  processors. Plugins based on TI \texttt{OpenMAX} IL multimedia interface.
  Maintainer of the camera and video encoder plugins.
\item \textbf{Embedded Software Engineer} (August 2006 -- August 2007)\\
  Software developer for the \texttt{OpenMAX} IL Camera component for
  \texttt{OMAP2} family processors.
\end{itemize}
\end{category}

\begin{category}{Academic \\experience}

\citem\itesm\\
\begin{itemize}
\item \textbf{Software Embedded} (August 2014 -- November 2014).\\
  A \textit{Embedded Linux Software} course imparted on the Department of
  Master of Electronic. Topics covered: cross-toolchain, bootloader,
  kernel, filesystems and multimedia through HW accelerators using
  \texttt{i.MX} processors.
\end{itemize}
\end{category}


\begin{category}{Research \\experience}

\citem\itesm\\
\begin{itemize}
\item \textbf{Web programmer} (Part time, August 2009 -- January 2011).\\
  Design and implementation of an Second factor authentication system using
  browser User Agent. A functional prototype built and delivered to customer.
\item \textbf{Web programmer} (Full time, August 2007 -- July 2008).\\
  Design and implementation of a classification method for detecting intruders
  using keystrokes typing rhythm. The designed algorithm documented and
  delivered successfully to customer (Google).
\end{itemize}

\citem{\inaoe}\\
\begin{itemize}
\item \textbf{Master thesis in Computer Astrophysics} (January 2001 -- December 2004).\\
  Comparison between Decay Times of Satellite Galaxies using $N$-body
  numerical simulations and a two-body motion equation with dynamical
  friction. Several galaxy density profiles were compared. Thesis partially done in Marseille, France\\
  Advisors:  Ivanio Puerari (INAOE, M\'exico) and Lia Athanassoula (Observatoire de Marseille, France).
\item \textbf{Scientific Programmer} (August -- December 2000).
  Comparison between two classification methods (\textsl{k}-nn and neural networks) using stellar spectra.\\
  Advisor: Olac Fuentes (INAOE, M\'exico).
\end{itemize}
\end{category}

% -------- Publication --------------------------------------------

\begin{category}{Talks}

\citembullet Simulation of Decay Times of Satellite Galaxies using Semi-Analytical methods.\\
Proceedings of The National Convention of Astronomy 2003.\\
Instituto de Astronom\'{\i}a, UNAM. \\
Authors: Leonardo Sandoval, Lia Athanassoula and Jorge Villa
\end{category}

% ------- Skills ------------------------------------------------------
\begin{category}{Skills}
\citembullet Programming languages: Python, shell (bash), C, X86 assembler
\citembullet Version controllers: Git
\citembullet Linux Building Systems: LTIB, Yocto Project
\end{category}

\begin{category}{Languages}
\citembullet Native Spanish
\citembullet Fluent spoken/written English
\citembullet Fair \textbf{German}. ZDaF (Zertifikat Deutsch als Fremdsprache) obtained in Goethe-Institut, Guadalajara, Jalisco, Mexico (1996-1998)
\end{category}

% ----- Hobbies ------------------------
\begin{category}{Hobbies}
\citembullet Playing Organ
\citembullet Tech and Spiritual book reading
\end{category}

% -------- Reference --------------------------------------------
\begin{category}{Reference} 
\citemnobullet Available on request.
\end{category}

\end{document}
